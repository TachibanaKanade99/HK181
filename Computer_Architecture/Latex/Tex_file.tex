\documentclass[10pt]{article}
\usepackage{pictex,amsmath,amsfonts,amssymb,amsthm,verbatim}
\usepackage{fullpage}
\usepackage{fullpage}
\usepackage{fancyhdr}
\usepackage{algorithm,algorithmic}
\usepackage{multirow}
\usepackage{gensymb}
\usepackage{mathrsfs}

\setlength{\voffset}{-0.25in}
\setlength{\headsep}{+0.5in}
\setlength{\parskip}{1em}
\setlength{\parindent}{0em}

\def\vu{\mathbf{u}}
\def\vv{\mathbf{v}}
\def\vb{\mathbf{b}}
\def\vw{\mathbf{w}}
\def\vs{\mathbf{s}}

\usepackage{xcolor}
\usepackage{titlesec}
\usepackage{mdframed}
\usepackage[utf8]{vietnam}

\newmdenv[linecolor=blue,skipabove=\topsep,skipbelow=\topsep,leftmargin=5pt,rightmargin=-5pt,innerleftmargin=5pt,innerrightmargin=5pt]{mybox}

\newcommand{\quotes}[1]{``#1''}
\usepackage{minted}
\usepackage{graphicx,graphics}

\begin{document}
	\begin{center}
		COMPUTER  ARCHITECTURE
	\end{center}

	\section{Chapter II: Instruction - Language of Computer:}
	\quotes{The world in language is \textbf{Instruction}, and its vocabulary is \textbf{Instructon set}} \\

	\subsection{Operation of the Computer Hardware: }
	- Every computer has to perform arithmetic. The MIPS assembly language notation: \\
	\begin{minted}{gas}
	add a, b, c
	\end{minted}

	\begin{itemize}
		\item This instruction will perform addition of b and c then put their sum in a.
		\item This notation is rigid in that each MIPS arithmetic instruction perform only one operation and must always have exactly 3 variables.

	\bigbreak
	\begin{minted}{gas}
	add a, b, c # The sum of b and c is placed in a:
	add a, a, d # The sum of b, c and d is placed in a:
	add a, a, e # The sum of b, c, d and e is placed in a:
	\end{minted}
	$\rightarrow{\mbox{ It take three instructions to perform instruction adds of 4 variables }}$

	\bigbreak
	\includegraphics[scale = 0.4]{hinh}
	\bigbreak
	\includegraphics[scale = 0.4]{hinh1} \\
	\includegraphics[scale = 0.4]{hinh2}
	\bigbreak

		\item Each line of Assembly language can contain at most 1 instruction.
		\item Comments always terminate at the end of the line
	\end{itemize}

	\begin{mybox}
		\begin{center}
			\textit{Design Priciple I:} Simplicity favors Regularity
		\end{center}
	\end{mybox}

	\textbf{Example: } \\
	\begin{center}
		\begin{minted}{cpp}
		f = (g + h) - (i + j)
		\end{minted}
	\end{center}
	\textbf{Solution: } \\
	At first the compiler would break this statement into several MIPS instructions: \\
	\begin{minted}{gas}
	add t0, g, h # tempory variable t0 contain g + h:
	add t1, i, j # tempory varialbe t1 contain j + j:
	\end{minted}
	Then we perform the subtraction: \\
	\begin{minted}{gas}
	sub f, t0, t1 # get t0 - t1 equals to (g + h) - (i + j):
	\end{minted}

	\subsection{Operand of the Computer Hardware: }
	\quotes{Unlike others high-level programming language, the operand of arithmetic instructions are restricted; they must be a limited number of special location built directly in hardware called \textit{register}}. \\

	- \textit{Registers} are primitives used in hardware design that are also visable to programmer when the computer is completed. \\
	- The size of \textit{a register} in the MIPS architecture is 32 bits. Groups of 32 bits occur so frequently that they are called \textbf{word}. \\
	- The major difference between variable and register is the limited number of registers, typically 32 on current computer like MIPS. \\

	\begin{mybox}
		\begin{center}
			\textit{Design Priciple II:} Smaller is faster
		\end{center}
	\end{mybox}

	- A very large number of registers may increase the clock cycle time supply because it take electric signals longer when they make travel farther. \\

	\subsection{Memory Operand: }
	\begin{itemize}
		\item Since arithmetic operations occur only in MIPS instructions, thus MIPS must include instructions that transfer data between memory and register. Such instructions are called \textbf{data transfer instructions}. 
		\item To access the word in memory, the instruction must suppply the memory \textbf{memory address}. \\
	 		\includegraphics[scale = 0.5]{hinh3.png}
			\bigbreak
		\item The data transfer instruction that copies data from memory to register is called \textbf{load}. The actual name for this instruction in MIPS is \textit{lw}, standing for \textit{load word}.
		\item In MIPS, words must start at address that are multiplies of 4. This requirement is called \textbf{alignment restriction}.
	\end{itemize}

	\includegraphics[scale = 0.5]{hinh4.png}

	\subsection{Constant or Immediate Operand: }
	- To use constant as an operand we woudl have to load a constant from memory to use: \\
	\begin{minted}{gas}
	lw $t0, AddrConstant4($s1) # $t0 = constant 4
	add $s3, $s3, $t0 # $s3 = $s3 + $t0 ($t0 = 4)
	\end{minted}

	- Alternative method to add constant is using \textit{add immediate} or \textit{ad}: \\
	\begin{minted}{gas}
	addi $s3, $s3, 4 # $s3 = s3 + 4
	\end{minted}

	\subsection{Signed and Unsigned Number: }
	- Number of bits within a MIPS word and the placement of the the number $1011_{two}$: \\
	\includegraphics[scale = 0.4]{hinh5.png}
	\bigbreak
	- The MIPS word is 32 bits long, so we can represent $2^{32}$ different 32-bit patterns. \\
	- The way to represent the signed and unsigned number is called \textit{two's complement} representation:
	\begin{itemize}
		\item The leading 0s represent for \textit{positive}
		\item THe leading 1s represent for \textit{negative}
	\end{itemize}
	\includegraphics[scale = 0.4]{hinh6.png} 
	\bigbreak
	\includegraphics[scale = 0.4]{hinh7.png}
	\bigbreak

	Ex:
	\includegraphics[scale = 0.3]{hinh8.png}
	\bigbreak

	\textbf{Revision: Binary Addition } \\
	\begin{tabular}{|c|c|c|c|}
	\hline 
	a & b & a + b & carry \\
	\hline 
	0 & 0 & 0 & 0 \\
	\hline
	0 & 1 & 1 & 0 \\
	\hline
	1 & 0 & 1 & 0 \\
	\hline
	1 & 1 & 0 & 1 \\
	\hline
	\end{tabular}

	






\end{document}                          